% LaTeX

\documentclass[12pt]{amsart} \usepackage{amssymb}

%\textwidth = 460pt 
%\textheight = 9in 
%\hoffset=-54pt \voffset=-40pt

% SIDE MARGINS:
\oddsidemargin 0in \evensidemargin 0in

% VERTICAL SPACING:
%\topmargin -.15in
\topmargin -.5in
\headheight 0in \headsep 0.0in
%\footheight 0.5in
\footskip 0.5in

\pagestyle{plain}
%\pagenumbering{}

% DIMENSION OF TEXT:
\textheight 10in \textwidth 6.7in
%

%\textwidth = 470pt
%\textheight = 700pt
%%\topmargin = 0pt
%%\oddsidemargin = 0pt
%\hoffset = -60pt
%\voffset = -50pt

%\input epsf \def\epsfsize#1#2{0.4#1\relax} \def\nl{\hfil\break}

%\renewcommand{\baselinestretch}{1.2}
%\def\Indent{\hskip .2in}

\def\labelitemi{--}

\title[]{Methods of Communication}

\author[]{Emilie Menard Barnard}

\begin{document}
\maketitle
\thispagestyle{empty}

As an educator, I recognize that my students come from different backgrounds with varied perspectives and abilities. Because of this, I carefully chose my methods of communication during class meetings and outside of class time so that they serve my students effectively and efficiently.

\subsection*{Class Meetings}
Note that I refer to lectures as ``class meetings." I strive to make class time more engaging than a generic span of time where I stand at the front of the class and lecture at students. My class meetings are generally structured in the following way: 
\begin{itemize}
\item 7-15 minutes of journaling
\item No more than 50 minutes of instruction
\item 10 minute break
\item Another 50 minutes or less of instruction
\end{itemize}

At the start of each class meeting, I provide a \textbf{journal} prompt for my students. This provides a quiet time for self-reflection and gives students time to mentally prepare for the upcoming content. I encourage students to respond to the prompts however they want --- words, charts, drawings, etcetera. Topics include rating their confidence in concepts covered in the previous class or assignment, current events in technology, and applying content in class to their everyday lives. Students keep their journals at the end of the class so that they may later on reflect back on their progress. I also use these for attendance and course evaluations.

When introducing new topics, I minimize the amount of time I do live-coding during class meetings. I find it is more important, especially in introductory courses, to cover topics sans computers as much as possible before diving into code, as students often jump into coding without actually thinking about the task at hand. If instead we focus on the concept itself first, the code will be easier to manage. I teach with \textbf{slides} on the projector screen as I bounce around the room, not with an IDE running the whole class period. These slides always have an outline at the beginning and a set of references at the end if students would like to learn more about the topics on their own time. My slides contain only key terms and basic code examples. I then provide \textbf{more detailed notes on the course website}, which enables students to focus more on understanding examples and less on writing notes. I prepare these online notes myself using more casual language rather than take them from formal textbook sources in an attempt to make them easier to digest. As a visual learner myself, I also frequently use the \textbf{board} to diagram concepts when appropriate.

Rather than just lecturing at the students the whole class period, I plan a few \textbf{interactive activities} for each class meeting. These can be simple ``think, pair, shares" or more complex activities tailored to the content. I use various types of activities to keep them interesting. Some require individual work, whereas others are done in small groups. Some require coding, some are completed with pen and paper. These activities not only break up a ``lecture" into more digestible chunks, but also provide me an opportunity to walk around and see how everyone is understanding the material individually.

When an activity is over and focus needs to switch to something new, I employ the ``if you can hear me clap once, if you can hear me clap twice" tactic, where I wait after each statement until the majority of the students are clapping in return. Interrupting a thought in the classroom is never something I like to do, but this method is at least a bit fun for the students as they get to make some noise.

I also \textbf{move around the classroom} as I teach, as this helps keeps students engaged. If a student raises their hand to ask a question, for example, I will walk over to them to hear their question. This way I will almost always hear the question the first time (and can repeat it louder for the class to hear if necessary), and the student avoids the seemingly-stressful spotlight associated with having to repeat a question.
\vspace{1cm}

\subsection*{Outside of Class}
Communication during class time is essential for introducing information, but investment outside of class time is necessary for fostering deeper learning. I usually arrive at the classroom 30 minutes early which allows students a chance to \textbf{chat with me casually} before class starts (and gives me some time to set up since projectors never work the first time you press the ``on" button). I have found that this helps to humanize me and shows students that computer science is not a mystical topic that only an elite few can understand. This also helps them feel more comfortable asking questions during class time as it removes the initial nerves associated with speaking up during class.

For even more options, \textbf{I host both traditional in-person and online office hours}. While I generally find that in-person office hours are more effective, some students prefer the online option and exclusively attend my online offerings. This is especially helpful for students with busy schedules, families, or introverts who simply prefer to talk to people via instant messaging from the comfort of their own home.

I am also extremely responsive to student \textbf{emails}. I respond to most within a few hours, but to all within 24 hours, with two exceptions: on weekends and when traveling. On my syllabi, I state that I will be generally less-responsive to emails over the weekend. I also take advantage of the \textbf{online forums} provided on the online course management system to notify students in advance if I will be less responsive to emails on a certain day (if traveling for a conference, for example).

\vspace{1cm}
I strive to provide many channels of communication for my students. Different types appeal to   different students, and my goal is for each student to feel comfortable reaching out to me somehow. Otherwise, their opportunity for learning is impaired. With this in mind, I  constantly seek ideas that will improve my methods of communication with students both during and outside of class time.


\end{document}