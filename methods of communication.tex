% LaTeX

\documentclass[12pt]{amsart} \usepackage{amssymb}

%\textwidth = 460pt 
%\textheight = 9in 
%\hoffset=-54pt \voffset=-40pt

% SIDE MARGINS:
\oddsidemargin 0in \evensidemargin 0in

% VERTICAL SPACING:
%\topmargin -.15in
\topmargin -.5in
\headheight 0in \headsep 0.0in
%\footheight 0.5in
\footskip 0.5in

\pagestyle{plain}
%\pagenumbering{}

% DIMENSION OF TEXT:
\textheight 10in \textwidth 6.7in
%

%\textwidth = 470pt
%\textheight = 700pt
%%\topmargin = 0pt
%%\oddsidemargin = 0pt
%\hoffset = -60pt
%\voffset = -50pt

%\input epsf \def\epsfsize#1#2{0.4#1\relax} \def\nl{\hfil\break}

%\renewcommand{\baselinestretch}{1.2}
%\def\Indent{\hskip .2in}

\def\labelitemi{--}

\title[]{Methods of Communication}

\author[]{Emilie Menard Barnard}

\begin{document}
\maketitle
\thispagestyle{empty}

As an educator, I recognize that my students come from different backgrounds with varied perspectives and abilities. Because of this, I carefully chose my methods of communication during class meetings and outside of class time so that they serve my students in the most effectively and efficiently.

\subsection*{Class Meetings}
Note that I instead refer to lectures as ``class meetings." I strive to make class time more engaging than a generic span of time where I stand at the front of the class and lecture at students My class meetings are generally structured in the following way: 
\begin{itemize}
\item 7-15 minutes of journaling
\item No more than 50 minutes of instruction
\item Mid-way 10 minute break
\item Another 50 minutes or less of instruction
\end{itemize}

At the start of each class meeting, I provide a \textbf{journal} prompt for my students. This provides a quiet time for self-reflection and gives students time to mentally prepare for the upcoming content. I encourage students to respond to the prompts however they want - words, charts, drawings, etc. Topics include rating their confidence in concepts covered in the previous class or assignment, current events in technology, and challenges to apply content in class to their everyday lives. Students are allowed to keep their journals at the end of the class so that they may later on reflect back on their progress. I also use these for attendance. 



%notes:
%
%all i can really come up with though is:
%
%1) office hours - I host both online and in person ones because I realize students sometimes don't like to come in person for various reasons (restricted schedule because of other responsibilities, some students prefer instant messaging to in person stuff, anxiety, etc)
%
%2) I come to the classroom about 30-45 minutes early partly because I have anxiety about the projector not working but also that way students can chat with me casually before class starts
%
%3) I respond to emails super fucking quickly. I have a OCD-like issue where I'm constantly checking email
%
%off the top of my head this is what I can think of:
%
%1) all course content and main topics I cover in class are available online so students don't have to worry about jotting all the notes down and can focus more on examples and smaller nuance-y things during class meetings
%
%2) i do online "homework quizzes" that correspond to each lecture so students can check their understanding before the next class meeting and I can see which topics I need to review at the start of the next lecture
%
%3) in addition to my slides which will have key terms written out, etc I use the whiteboard/chalkboard during class to do a lot of visuals
%
%4) i minimize the amount of time I do live-coding at the lecture podium. I find it's more important to cover topics sans computers as much as possible before you dive into coding stuff because too often students jump into coding the thing without thinking about the thing, so i do as much teaching as possible without the actual development environment (the software where you write the code). i do of course spend some class time showing how the development environment works but i don't stand up there and code for all my subsequent lectures. watching someone code is boring af.
%
%5) I do interactive shit. like "try typing this in now and see what happens, why do you think that happened? why didn't it do this instead?" also some group stuff like think pair share, etc
%
%6) I start each lecture with journal entries. i give students a prompt (either about current events in tech, topics we learned about in class, or general "how can you apply this to your life" stuff) and 10 mins to respond. it's also how i take attendance and deal with students inevitably showing up late.



\subsection*{Outside of Class}
loremipsum

\vspace{1cm}
add improvement bit at the end\\
Because of this, I am constantly working on improving my methods of communication with students, both during class meetings and outside of class time.

%\section*{Background}
%\noindent I knew nothing about computer science before I started university. After taking a few core courses, I started to realize some disadvantages I had as a student in the computer science program. Many of my peers' parents not only had college degrees, but also held degrees in computer science or related fields. They were able to discuss their projects and new course concepts learned with their families. I, however, have not been able to discuss my courses with my parents since the 5th grade. Neither of my parents have ever programmed a computer, or even attended college. Gender differences set me further apart from my peers. I will always remember the discrimination I faced in one of my sophomore-year courses. My project group was the first female-only group our professor had seen, which seemed to give the professor and my peers doubts that our group could complete our project. I was shocked by the comments in class, and can only imagine the content of the comments that I did not hear.

%\section*{Why I Teach}
%\noindent My goal as a computer science educator is to craft a welcoming and productive environment for my students to explore the field. \textbf{I want everyone to feel like they can study and learn computer science if they are interested, no matter their background or previous experience.} I firmly believe that all students can excel in the classroom (and beyond) with an effective teaching style, proper encouragement, and enough self-motivation.


%\section*{My Current Goals for Growth}
%\noindent I am looking forward to accepting the responsibilities of an official course instructor again. I am hopeful that my fresh perspective will improve each course I am able to teach, and am excited to see how my approach helps students.
%\\\\\noindent I am also interested in teaching more varied courses. I currently have significant experience teaching introductory courses using the Python programming language, but I am eager to teach other types of courses as well.
%\\\\\noindent In addition, I hope to support and encourage students outside of the classroom. I am interested in service positions in addition to lectureship appointments.
%
%\section*{Conclusion}
%\noindent I am passionate about academics, schooling, and the learning process. I am excited to continue to give back, especially in more official educator roles. Most importantly, I want to be an example for the many students who, like myself, have been in disadvantaged and isolated positions. I want them to know that they too can pursue higher education and succeed in their academics, especially in a discipline as demanding as computer science.
\end{document}
