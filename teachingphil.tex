% LaTeX

\documentclass[12pt]{amsart} \usepackage{amssymb}

%\textwidth = 460pt 
%\textheight = 9in 
%\hoffset=-54pt \voffset=-40pt

% SIDE MARGINS:
\oddsidemargin 0in \evensidemargin 0in

% VERTICAL SPACING:
%\topmargin -.15in
\topmargin -.4in
\headheight 0in \headsep 0.0in
%\footheight 0.5in
\footskip 0.5in

\pagestyle{plain}
%\pagenumbering{}

% DIMENSION OF TEXT:
\textheight 10in \textwidth 6.5in
%

%\textwidth = 470pt
%\textheight = 700pt
%%\topmargin = 0pt
%%\oddsidemargin = 0pt
%\hoffset = -60pt
%\voffset = -50pt

%\input epsf \def\epsfsize#1#2{0.4#1\relax} \def\nl{\hfil\break}

%\renewcommand{\baselinestretch}{1.2}
%\def\Indent{\hskip .2in}

\def\labelitemi{--}

\title[]{Statement of Teaching Philosophy}

\author[]{Emilie Menard Barnard}

\begin{document}
\maketitle
\thispagestyle{empty}

\section*{Background}
\noindent I knew nothing about computer science before I started university. After taking a few core courses, I started to realize some disadvantages I had as a student in the computer science program. Many of my peers' parents not only had college degrees, but also held degrees in computer science or related fields. They were able to discuss their projects and new course concepts learned with their families. I, however, have not been able to discuss my courses with my parents since the 5th grade. Neither of my parents have ever programmed a computer, or even attended college. Gender differences set me further apart from my peers. I will always remember the discrimination I faced in one of my sophomore-year courses. My project group was the first female-only group our professor had seen, which seemed to give the professor and my peers doubts that our group could complete our project. I was shocked by the comments in class, and can only imagine the content of the comments that I did not hear.

\section*{Why I Teach}
\noindent With my background, I believe I relate to many students who are just starting to learn computer science. I understand first-hand how difficult school can be. I want to be an example for my students. I want to not only tell them, but also show them: ``Hey, I know it’s hard, but you can do this too!" I want all people to feel like they can study and learn computer science if they are interested, despite any setbacks. I firmly believe all students can excel in the classroom with an effective teaching style, encouragement, and enough self-motivation.

\section*{How I Teach}
\noindent As an educator, I strive to:
\begin{itemize}
\item \emph{Be available}: When a student has a question, I do everything within my ability to answer it within a timely manner. This ensures that there are fewer halts in the learning process.\vspace{2mm}
\item \emph{Be approachable}: In addition to being available, I also need to be approachable. Students need to feel comfortable approaching me with questions, ideas, or even just to discuss general topics. This fosters a better learning environment by encouraging questions and strengthening the relationships with my students.\vspace{2mm}
\item \emph{Encourage autonomy}: This is extremely important to teach in the introductory courses, so that students are better prepared for advance coursework. For instance, when a student asks a coding question that can easily be found in language specifications, such as ``how do I make a new turtle object in Python?" I encourage them to use an online search engine to find the answer, and I will give them hints for an efficient search query. By doing so, I help them learn how to help themselves so they don’t feel dependent on me to answer all questions. \vspace{2mm}
\newpage
\item \emph{Be inclusive}: Students like to discuss the difficulties of school and coursework. Sometimes they even suggest that they are intimidated by other students’ background experience. If they feel comfortable sharing this with me, I am succeeding in being an approachable teacher. I take this time to tell them that it is normal to feel this way, and that I myself even feel this way sometimes. I encourage them to not give up and remind them that what matters is what they get out of the class, not the other students. This is extremely important to remember in a field like computer science, where impostor syndrome is common. As a woman in computer science, I am also sensitive to gender-equality issues, and am more than willing to remind students to treat all other students with respect, if needed.\vspace{2mm}
\end{itemize}
In short, I want my students to feel like they can succeed, and actually succeed if they want to and put in the necessary effort. I want them to feel this way despite any setbacks, and I want them to be excited about learning so that the learning process is a positive experience.

\section*{My Current Goals for Growth}
\noindent I am looking forward to accepting the responsibilities of an official course instructor again. I am hopeful that my fresh perspective will improve each course I am able to teach, and am excited to see how my approach helps students.
\\\\\noindent I am also interested in teaching more varied courses. I currently have significant experience teaching introductory courses using the Python programming language, but I am eager to teach other types of courses as well.
\\\\\noindent In addition, I hope to support and encourage students outside of the classroom. I am interested in service positions in addition to lectureship appointments.

\section*{Conclusion}
\noindent I am passionate about academics, schooling, and the learning process. I am excited to continue to give back, especially in more official educator roles. Most importantly, I want to be an example for the many students who, like myself, have been in disadvantaged and isolated positions. I want them to know that they too can pursue higher education and succeed in their academics, especially in a discipline as demanding as computer science.
\end{document}
